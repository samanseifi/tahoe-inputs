%% This is file `elsarticle-template-2-harv.tex',
%%
%% Copyright 2009 Elsevier Ltd
%%
%% This file is part of the 'Elsarticle Bundle'.
%% ---------------------------------------------
%%
%% It may be distributed under the conditions of the LaTeX Project Public
%% License, either version 1.2 of this license or (at your option) any
%% later version.  The latest version of this license is in
%%    http://www.latex-project.org/lppl.txt
%% and version 1.2 or later is part of all distributions of LaTeX
%% version 1999/12/01 or later.222222
%%
%% The list of all files belonging to the 'Elsarticle Bundle' is
%% given in the file `manifest.txt'.
%%
%% Template article for Elsevier's document class `elsarticle'
%% with harvard style bibliographic references
%%
%% $Id: elsarticle-template-2-harv.tex 155 2009-10-08 05:35:05Z rishi $
%% $URL: http://lenova.river-valley.com/svn/elsbst/trunk/elsarticle-template-2-harv.tex $
%%
%%\documentclass[preprint,authoryear,12pt]{elsarticle}

%% Use the option review to obtain double line spacing
%% \documentclass[authoryear,preprint,review,12pt]{elsarticle}

%% Use the options 1p,twocolumn; 3p; 3p,twocolumn; 5p; or 5p,twocolumn
%% for a journal layout:
%% \documentclass[final,authoryear,1p,times]{elsarticle}
%% \documentclass[final,authoryear,1p,times,twocolumn]{elsarticle}
%% \documentclass[final,authoryear,3p,times]{elsarticle}
\documentclass[final,authoryear,3p,times,twocolumn]{elsarticle}
%% \documentclass[final,authoryear,5p,times]{elsarticle}
%% \documentclass[final,authoryear,5p,times,twocolumn]{elsarticle}

%% if you use PostScript figures in your article
%% use the graphics package for simple commands
%% \usepackage{graphics}
%% or use the graphicx package for more complicated commands
%% \usepackage{graphicx}
%% or use the epsfig package if you prefer to use the old commands
%% \usepackage{epsfig}

%% The amssymb package provides various useful mathematical symbols
\usepackage{amssymb}
\usepackage{epstopdf}
\usepackage{lmodern}
\usepackage[latin9]{inputenc}
\usepackage[T1]{fontenc}
\usepackage{color}
\usepackage{float}
\usepackage{tabularx}
\newcommand{\mbf}{\mathbf}
%% The amsthm package provides extended theorem environments
%% \usepackage{amsthm}

%% The lineno packages adds line numbers. Start line numbering with
%% \begin{linenumbers}, end it with \end{linenumbers}. Or switch it on
%% for the whole article with \linenumbers after \end{frontmatter}.
%% \usepackage{lineno}

%% natbib.sty is loaded by default. However, natbib options can be
%% provided with \biboptions{...} command. Following options are
%% valid:

%%   round  -  round parentheses are used (default)
%%   square -  square brackets are used   [option]
%%   curly  -  curly braces are used      {option}
%%   angle  -  angle brackets are used    <option>
%%   semicolon  -  multiple citations separated by semi-colon (default)
%%   colon  - same as semicolon, an earlier confusion
%%   comma  -  separated by comma
%%   authoryear - selects author-year citations (default)
%%   numbers-  selects numerical citations
%%   super  -  numerical citations as superscripts
%%   sort   -  sorts multiple citations according to order in ref. list
%%   sort&compress   -  like sort, but also compresses numerical citations
%%   compress - compresses without sorting
%%   longnamesfirst  -  makes first citation full author list
%%
%% \biboptions{longnamesfirst,comma}

% \biboptions{}

\journal{Journal of the Mechanics and Physics of Solids}

%---------------------- editing macros ------------------
\newcommand{\hsp}[1]{\textcolor{red}{[\textit{hsp}: #1]}}
\newcommand{\sss}[1]{\textcolor{blue}{[\textit{ss}: #1]}}
\newcommand{\hspi}[1]{\textcolor{red}{#1}}
\newcommand{\hsps}[1]{\textcolor{red}{\sout{#1}}}
%---------------------- editing macros ------------------

\begin{document}

\begin{frontmatter}

%% Title, authors and addresses

%% use the tnoteref command within \title for footnotes;
%% use the tnotetext command for the associated footnote;
%% use the fnref command within \author or \address for footnotes;
%% use the fntext command for the associated footnote;
%% use the corref command within \author for corresponding author footnotes;
%% use the cortext command for the associated footnote;
%% use the ead command for the email address,
%% and the form \ead[url] for the home page:
%%
%% \title{Title\tnoteref{label1}}
%% \tnotetext[label1]{}
%% \author{Name\corref{cor1}\fnref{label2}}
%% \ead{email address}
%% \ead[url]{home page}
%% \fntext[label2]{}
%% \cortext[cor1]{}
%% \address{Address\fnref{label3}}
%% \fntext[label3]{}

\title{Computational Modeling of Electro-Elasto-Capillary Phenomena in Dielectric Elastomers}

%% use optional labels to link authors explicitly to addresses:
%% \author[label1,label2]{<author name>}
%% \address[label1]{<address>}
%% \address[label2]{<address>}

\author[ss]{Saman Seifi}
\ead[ss]{samansei@bu.edu}
\address[ss]{Department of Mechanical Engineering, Boston University, Boston, MA 02215}
\author[hsp]{Harold S. Park}
\ead[hsp]{parkhs@bu.edu}
\address[hsp]{Department of Mechanical Engineering, Boston University, Boston, MA 02215}

\begin{abstract}

We present a new finite deformation, dynamic finite element model of dielectric elastomers that incorporates surface tension to capture elastocapillary effects on the electromechanical deformation.  

\end{abstract}

\begin{keyword}
%% keywords here, in the form: keyword \sep keyword
dielectric elastomer \sep elastocapillary
%% MSC codes here, in the form: \MSC code \sep code
%% or \MSC[2008] code \sep code (2000 is the default)

\end{keyword} 

\end{frontmatter}

% \linenumbers

%% main text
\section{Introduction} 

\footnotetext{\textit{$^{a}$Department of Mechanical Engineering, Boston University, Boston, MA 02215, USA. E-mail:  samansei@bu.edu}}

\footnotetext{\textit{$^{\ast}$\textit{$^{b}$}~Department of Mechanical Engineering, Boston University, Boston, MA 02215, USA.  Fax:  001 617 353 5866; Tel:  001 617 353 4208; E-mail:  parkhs@bu.edu}}

Dielectric elastomers (DEs) have attracted significant attention in recent years as a soft and flexible actuation material~\citep{carpiSCIENCE2010,brochuMRC2010,biddissMEP2008}.  The salient characteristic of DEs is that if sandwiched between two compliant electrodes that apply voltage across its thickness, the DE can exhibit both significant thinning and in-plane expansion, where the in-plane expansion can often exceed several hundred percent~\citep{keplingerSM2012}.  The ability to undergo such large deformations has led to DEs being studied for both actuation-based applications, including artificial muscles and flexible electronics, and also for generation-based applications and energy harvesting~\citep{carpiSCIENCE2010,brochuMRC2010,mirMT2007}.

Detailed studies on the mechanics of DEs began about 15 years ago with the seminal experimental work of~\citet{pelrineSAA1998,pelrineSCIENCE2000}.  Since then, there have been many experimental~\citep{foxJMPS2008,keplingerPNAS2010,kofodJIMSS2003,kofodSAA2005,peiSPIE2004,planteIJSS2006,planteSAA2007,planteSMS2007,schlaakSPIE2005,wisslerSAA2007a,zhangSPIE2004,chibaSPIE2008,wangPRL2011a,wangAM2012,wangNC2012}, theoretical~\citep{suoJMPS2008,suoAMSS2010,goulbourneJAM2005,dorfmannAM2005,dorfmannJE2006,mcmeekingJAM2005,patrickSAA2007,planteSAA2007,planteSMS2007,wisslerSMS2005}, and recently a small number of computational studies~\citep{parkIJSS2012,parkSM2013,parkCMAME2013,zhouIJSS2008,zhaoAPL2007,vuIJNME2007,wisslerSMS2005,buschelIJNME2013,khanCM2013,henannJMPS2013,liSMS2012} aimed that identifying the mechanisms that have the largest impact on the nonlinear dynamical behavior and failure mechanisms of DEs.  A summary of recent developments on electromechanical instabilities in DEs has been given by~\citet{zhaoAPR2014}.  

Concurrently, researchers have for many years studied the forces exerted by fluids at rest upon solids, which are known as surface tension, or elastocapillary forces.  While the best-known example of surface tension is likely that of deforming liquid droplets, there has been interest in using it to deform solid structures, see the reviews of~\citet{romanJPCM2010} and~\citet{liuAMS2012}.  Specifically, there has recently been interest in using elastocapillary forces to deform \emph{soft} structures in controllable or unique ways, since for these systems the elastocapillary number, which is defined as $\gamma/\mu l$, where $\gamma$ is the surface tension, $\mu$ is the shear modulus and $l$ is a characteristic length, is close to unity, implying that elastocapillary effects can be substantial for these soft materials.  

While elastocapillary effects have been extensively studied in soft materials, its effect on soft materials like DEs that deform when subject to an electric field, is a relatively unknown phenomenon.  For example,~\citet{wangPRE2013} performed interesting experiments of electrostatically deforming a constrained DE in a liquid solution, and showed that the instability mechanism of the surface could be tuned depending on the value of the surface tension.  Furthermore,~\citet{pineiruaSM2010} showed how elastocapillary origami could be developed by coupling surface tension with electric fields to deform a liquid droplet surrounded by a thin sheet of PDMS.  Overall, this discussion makes clear that there may be potential in using surface tension as an additional degree of freedom to introduce new and interesting deformation mechanisms in electroactive polymers like DEs, and furthermore that the computational tools needed to investigate such phenomena are currently lacking.  

Therefore, the objective of the present work is to present a finite element (FE) model for DEs accounting for the effects of surface tension, such that electro-elasto-capillary phenomena in DEs can be computationally investigated.  We pay particular interest to those instances where surface tension couples to and impacts known electromechanical instabilities that dielectric elastomers are known to undergo, specifically snap-through instability~\citep{pelrineSCIENCE2000} and surface creasing and wrinkling~\citep{wangPRL2011,wangAM2012,wangPRE2013}.
ion as an additional degree of freedom to introduce new and interesting deformation mechanisms in electroactive polymers like DEs, and furthermore that the computational tools needed to investigate such phenomena are currently lacking.  

Therefore, the objective of the present work is to present a finite element model for DEs accounting for the effects of surface tension, such that electro-elasto-capillary phenomena in DEs can be computationally investigated.  We pay particular interest to those instances where surface tension couples to and impacts known electromechanical instabilities that dielectric elastomers are known to undergo, specifically snap-through instability~\citep{pelrineSCIENCE2000}, surface creasing~\citep{wangPRL2011,wangAM2012}, and bursting drops leading to crack-like propagation~\citep{wangNC2012}.

\section{Background: Nonlinear Electromechanical Field Theory}

The numerical results we present in this work are based upon a finite element (FEM) discretization of the electromechanical field theory recently proposed by~\citet{suoJMPS2008}, and recently reviewed by~\citet{suoAMSS2010}.   In this field theory at mechanical equilibrium, the nominal stress $S_{iJ}$ satisfies the following (weak) equation:
\begin{equation}\label{eq:suo1} \int_{V}S_{iJ}\frac{\partial\xi_{i}}{\partial X_{J}}dV=\int_{V}\left(B_{i}-\rho\frac{\partial^{2}x_{i}}{\partial t^{2}}\right)\xi_{i}dV+\int_{A}T_{i}\xi_{i}dA,
\end{equation}
where $\xi_{i}$ is an arbitrary vector test function, $B_{i}$ is the body force per unit reference volume $V$, $\rho$ is the mass density of the material and $T_{i}$ is the force per unit area that is applied on the surface $A$ in the reference configuration.  

For the electrostatic problem, the nominal electric displacement $\tilde{D}_{I}$ satisfies the following (weak) equation:
\begin{equation}\label{eq:suo2} -\int_{V}\tilde{D}_{I}\frac{\partial\eta}{\partial X_{I}}dV=\int_{V}q\eta dV+\int_{A}\omega\eta dA,
\end{equation}
where $\eta$ is an arbitrary scalar test function, $q$ is the volumetric charge density and $\omega$ is the surface charge density, both with respect to the reference configuration.  It can be seen that the strong form of the mechanical weak form in (\ref{eq:suo1}) is the momentum equation, while the strong form of the electrostatic weak form in (\ref{eq:suo2}) is Gauss's law.  

As the governing field equations in (\ref{eq:suo1}) and (\ref{eq:suo2}) are decoupled, the electromechanical coupling occurs through the material laws.  The hyperelastic material law we adopt here has been utilized in the literature to study the nonlinear deformations of electrostatically actuated polymers; see the works of~\citet{vuIJNME2007}, and~\citet{zhaoAPL2007}.  Due to the fact that the DE is a rubber-like polymer, phenomenological free energy expressions are typically used to model the deformation of the polymer chains.  In the present work, we will utilize the form~\citep{vuIJNME2007,zhaoAPL2007}
\begin{equation}\label{eq:de1} W(\mbf{C},\tilde{\mbf{E}})=\mu W_{0}-\frac{1}{2}\lambda(\ln{J})^{2}-2\mu W_{0}(3)\ln{J}-\frac{\epsilon}{2}JC_{IJ}^{-1}\tilde{E}_{I}\tilde{E}_{J},
\end{equation}
where $W_{0}$ is the mechanical free energy density in the absence of an electric field, $\epsilon$ is the permittivity, $J=\det(\mbf{F})$, where $\mbf{F}$ is the continuum deformation gradient, $C_{IJ}^{-1}$ are the components of the inverse of the right Cauchy-Green tensor $\mbf{C}$, $\lambda$ is the bulk modulus and $\mu$ is the shear modulus.  

We model the mechanical behavior of the DE using the Arruda-Boyce rubber hyperelastic function~\citep{arrudaJMPS1993}, where the mechanical free energy $W_{0}$ in (\ref{eq:de1}) is approximated by the following truncated series expansion,
\begin{eqnarray}\label{eq:de2} W_{0}(I_{1})=\frac{1}{2}(I_{1}-3)+\frac{1}{20N}(I_{1}^{2}-9)+\frac{11}{1050N^{2}}(I_{1}^{3}-27) \\ \nonumber
+\frac{19}{7000N^{3}}(I_{1}^{4}-81)+\frac{519}{673750N^{4}}(I_{1}^{5}-243),
\end{eqnarray}
where $N$ is a measure of the cross link density, $I_{1}$ is the trace of $\mbf{C}$, and where the Arruda-Boyce model reduces to a Neo-Hookean model if $N\rightarrow\infty$.  We note that previous experimental studies of~\citet{wisslerSAA2007a} have validated the Arruda-Boyce model as being accurate for modeling the large deformation of DEs.  

\section{Finite Element Formulation}
\subsection{Nonlinear, Dynamic Finite Element Model}

The FEM model we use was previously developed by~\citet{parkIJSS2012}.  In that work, the corresponding author and collaborators developed a nonlinear, dynamic finite element (FEM) formulation of the governing nonlinear electromechanical field equations of Suo \emph{et al.}~\cite{suoJMPS2008} that are summarized in (\ref{eq:suo1}) and (\ref{eq:suo2}).  By using a standard Galerkin FEM approximation to both the mechanical displacement and electric potential fields, and incorporating inertial effects in the mechanical momentum equation, an implicit, coupled, monolithic nonlinear dynamic FEM formulation was obtained with the governing equations~\citep{parkIJSS2012}
\begin{equation}\label{eq:fe1} \left(\begin{array}{cc}{\Delta\mbf{a}} \\ {\Delta\mbf{\Phi}}\end{array}\right)=-\left(\begin{array}{cc} {\mbf{M}+\beta\Delta t^{2}\mbf{K}^{mm}} & {\mbf{K}^{me}} \\ {\beta\Delta t^{2}\mbf{K}^{em}} & {\mbf{K}^{ee}} \end{array}\right)\left(\begin{array}{cc}{\mbf{R}^{mech}} \\ {\mbf{R}^{elec}}\end{array}\right)
\end{equation}
where $\Delta\mbf{a}$ is the increment in mechanical acceleration, $\Delta\Phi$ is the increment in electrostatic potential, $\beta=0.25$ is the standard Newmark time integrator parameter, $\mbf{R}^{mech}$ is the mechanical residual, $\mbf{R}^{elec}$ is the electrical residual, and the various stiffness matrices $\mbf{K}$ include the purely mechanical ($\mbf{K}^{mm}$), mixed electromechanical ($\mbf{K}^{me}=\mbf{K}^{em}$), and purely electrostatic ($\mbf{K}^{ee}$) contributions.  Details regarding the residual vectors and the various mechanical, electromechanical and electrostatic stiffnesses can be found in previous work~\citep{parkIJSS2012}.  

In the present work, volumetric locking due to the incompressible material behavior was alleviated using the Q1P0 method of~\citet{simoCMAME1985}.  While viscoelastic effects have previously been accounted for within the FEM model~\citep{parkSM2013}, these effects are neglected in the present work such that the effects of surface tension on electromechanical instabilities in DEs can be investigated without other physical complications.

\subsection{Surface Tension}

Finite element formulations of surface tension have been given by~\citet{saksonoCM2006} and~\cite{javiliCMAME2010}, amongst others.  We note that the recently published work of~\citet{henannSM2014} uses the model presented by~\citet{saksonoCM2006}, with an incompressible hyperelastic material model for the deforming solid.  In the present work, we utilize the dynamic formulation of~\citet{saksonoCM2006a}, where the utility and importance of using inertia to capture, using FEM, the electromechanical instabilities that occur in DEs was shown in the previous works of~\citet{parkIJSS2012,parkCMAME2013,parkSM2013}.  We now briefly describe the formulation, and the resulting electro-elasto-capillary FEM equations that we solve, while a schematic of the current and reference configurations is shown in Figure (\ref{ref}).

\begin{figure} \begin{center}
\includegraphics[scale=0.4]{pics/refdef.pdf}
\caption{The reference body $B_0$ and the deformed body $B_t$ at time $t$.}
\label{ref} \end{center}
\end{figure}

Using the~\citet{saksonoCM2006a} model, we begin with the force continuity at the solid-liquid interface, 
\begin{equation}\label{eq:sak1} \mbf{\sigma n}=-p_{ext}\mbf{n}+2H\gamma\mbf{n}
\end{equation}
where $\mbf{n}$ is the unit normal, $\sigma$ is the Cauchy stress, $H$ is the mean curvature, and $p_{ext}$ is an external pressure.  The spatial (updated Lagrangian) form of the weak form of the momentum equation can then be written as
\begin{eqnarray}\label{eq:sak2} \int (\sigma:\nabla\mbf{w}-\rho(\mbf{b-a})\cdot\mbf{w})dv-\int \mbf{t}\cdot\mbf{w}da-\int (-p_{ext}\mbf{n}\cdot\mbf{w})da \\ \nonumber
+\int (\gamma\nabla_{s}\cdot\mbf{w})da-\int (\gamma\mbf{w}\cdot\mbf{m})ds = 0
\end{eqnarray}
where $\mbf{w}$ is the virtual displacement, $\mbf{t}$ is the traction, $\mbf{b}$ is the body force, $\nabla_{s}=(\mbf{I}-\mbf{n}\otimes\mbf{n})\nabla$ is the surface gradient operator, $\rho$ is the density and $\mbf{a}$ is the acceleration.  In the present work, we neglect the three-phase contact line, i.e. the integral in (\ref{eq:sak2}) over the line $ds$ for simplicity.

The FEM form of (\ref{eq:sak2}) can be obtained by making the usual Galerkin approximation of both the displacements and virtual displacements with the same shape functions, resulting in the (mechanical) residual $\mbf{R}^{mech}(\mbf{X})$
\begin{equation}\label{eq:sak3} \mbf{R}^{mech}(\mbf{X})=\mbf{M}\ddot{\mbf{X}}+\mbf{F}^{int}-\mbf{F}^{ext}+\mbf{F}^{surf}=0
\end{equation}
where the various terms in (\ref{eq:sak3}) take the following form for each FE $e$
\begin{eqnarray}\label{eq:sak4} \mbf{M}_{e}=\int\rho\mbf{N}^{T}\mbf{N}dv \\ \nonumber
\mbf{F}_{e}^{int}=\int\mbf{B}^{T}\sigma dv \\ \nonumber
\mbf{F}_{e}^{ext}=\int\mbf{N}^{T}\mbf{b}dv + \int\mbf{N}^{T}\mbf{t}da \\ \nonumber
\mbf{F}_{e}^{surf}=\int\gamma\nabla_{s}\mbf{N}da
\end{eqnarray}
where $\mbf{N}$ and $\mbf{B}$ are the standard FEM shape function and gradient, respectively.  As can be seen in (\ref{eq:sak3}), the only non-standard term compared to the standard discretization of the momentum equation arises in the surface force $\mbf{F}^{surf}$, and the corresponding surface stiffness $\mbf{K}^{surf}$.  

~\citet{henannSM2014} analytically derived the surface force $\mbf{F}^{surf}$ for a two-dimensional, 4-node bilinear quadrilateral element as
\begin{equation}\label{eq:sak5} \mbf{F}^{surf}=-\frac{h\gamma}{L_{e}}\left(\begin{array}{cccc} {x_{1}-x_{2}} \\ {y_{1}-y_{2}} \\ {x_{2}-x_{1}} \\ {y_{2}-y_{1}} \end{array}\right)
\end{equation} 
The surface stiffness $\mbf{K}^{surf}$ can be obtained through linearization of the surface force $\mbf{F}^{surf}$ in (\ref{eq:sak3}).  This value for a 4-node bilinear quadrilateral element in two-dimensions was also given analytically by~\cite{henannSM2014} as
\begin{equation}\label{eq:sak6} \mbf{K}^{surf}=\frac{h\gamma}{L_{e}}\left(\begin{array}{cccc} {1} & {0} & {-1} & {0} \\ {0} & {1} & {0} & {-1} \\ {-1} & {0} & {1} & {0} \\ {0} & {-1} & {0} & {1} \end{array}\right)-\frac{h\gamma}{L_{e}^{3}}\left(\begin{array}{cccc} {x_{1}-x_{2}} \\ {y_{1}-y_{2}} \\ {x_{2}-x_{1}} \\ {y_{2}-y_{1}} \end{array}\right)\left(\begin{array}{cccc} {x_{1}-x_{2}} \\ {y_{1}-y_{2}} \\ {x_{2}-x_{1}} \\ {y_{2}-y_{1}} \end{array}\right)^{T}
\end{equation}
where $L_{e}$ is the length of the FE face containing nodes 1 and 2.  The ordering of the FE nodes corresponding to Eqs. (\ref{eq:sak5}) and (\ref{eq:sak6}) is shown in Fig. (\ref{element}).  

\begin{figure} \begin{center}
\includegraphics[scale=0.3]{pics/element.pdf}
\caption{Schematic of nodal numbering consistent with surface tension formulation given in Eqs. (\ref{eq:sak5}) and (\ref{eq:sak6}).  Nodes 1 and 2 are surface nodes.}
\label{element} \end{center}
\end{figure}

The final coupled electromechanical FEM equations we solve, which include the surface tension terms, can be written as
\begin{equation}\label{eq:sak7} \left(\begin{array}{cc}{\Delta\mbf{a}} \\ {\Delta\mbf{\Phi}}\end{array}\right)=-\left(\begin{array}{cc} {\mbf{M}+\beta\Delta t^{2}(\mbf{K}^{mm}+\mbf{K}^{surf})} & {\mbf{K}^{me}} \\ {\beta\Delta t^{2}\mbf{K}^{em}} & {\mbf{K}^{ee}} \end{array}\right)\left(\begin{array}{cc}{\mbf{R}^{mech}} \\ {\mbf{R}^{elec}}\end{array}\right)
\end{equation}
In comparing the new FE formulation including surface tension in (\ref{eq:sak7}) to the previous FEM equations of~\citet{parkIJSS2012} in (\ref{eq:fe1}), the only changes are the addition of the surface stiffness $\mbf{K}^{surf}$ to the standard mechanical stiffness matrix $\mbf{K}^{mm}$, as well as the surface contribution $\mbf{F}^{surf}$ to the mechanical residual $\mbf{R}^{mech}$, as shown previously in (\ref{eq:sak3}) and (\ref{eq:sak4}).

\section{Numerical Results}

All numerical simulations were performed using the open source code~\citet{tahoe} with standard 4-node, bilinear quadrilateral finite elements within a two-dimensional, plane strain approximation.  Before any application of electrostatic loading via applied charges or voltages, the surface tension is first applied incrementally until the desired value is reached; the incremental approach is necessary to avoid computational instabilities, as previously discussed by~\citet{javiliCMAME2010}.  The surface tension is applied incrementally by defining a target value of surface tension $\gamma$ we define the following function $\gamma_{0}$: 
\begin{equation} \gamma_{0}=min\left(\gamma,\frac{\gamma t}{t_0}\right)
\label{eq:gammaramp}
\end{equation}
where $t$ is the current time and $t_0$ is the total time allotted to reach the prescribed value for surface tension $\gamma$.  Once the system is in equilibrium with the surface tension, voltage or charge is applied in a monotonically increasing fashion.

\subsection{Single Element Tests}

We first perform a suite of parametric benchmark studies to gain a qualitative understanding of how surface tension impacts the electromechanical behavior of DEs undergoing homogeneous and inhomogeneous deformation.  A single 4-node bilinear quadrilateral element of unit length and height is used for these simulations.  For all numerical simulations, we chose the following constitutive parameters for the Arruda-Boyce model in Eq. (\ref{eq:de2}):  $\mu=\epsilon=1$, $\lambda=1000$ and $N=5.0$, while different values of the surface tension $\gamma$ are chosen.    

\subsubsection{Homogeneous Deformation}

\begin{figure} \begin{center} 
\includegraphics[scale=0.25]{pics/homo3.pdf}
\caption{(a) Schematic of single element with homogeneous boundary conditions; (b) Deformed configuration with $\gamma=0$; (c) Deformed configuration with $\gamma=10$.}
\label{homo3} \end{center} \end{figure}

The first set of boundary conditions are to allow homogeneous deformation of the DE. The electromechanical boundary conditions are as shown in Fig. (\ref{homo3})(a), with rollers on the -x and -y surfaces, and voltage applied at the top surface of the single square element.  The resulting configurations for surface tension values of $\gamma=0$ and $\gamma=10$ are shown in Figures (\ref{homo3})(b) and (c).  As can be seen, Figure (\ref{homo3})(b) shows the well-known deformed configuration where the DE contracts along the thickness direction while simultaneously elongating in response to the applied voltage.  However, the configuration in Figure (\ref{homo3})(c) is quite different when $\gamma=10$.  Instead of resulting in a rectangular deformed configuration, the single element takes a deformed configuration that does not appear much changed from the initial, square configuration, which shows that the impact of surface tension is to resist the deformation that would otherwise occur due to the applied voltage.

We plot the resulting normalized voltage-charge curves in Figure (\ref{homo}) for various values of $\gamma$.  We can see that the effect of increasing surface tension is to increase the voltage that is required to induce the electromechanical instability.  These results are quantitatively in agreement with previous experimental and theoretical studies on soft materials which found that the effect of surface tension is to create a barrier to instability nucleation~\citep{chenPRL2012,moraSM2011}.  We also note that a softening response is not observed in the voltage-charge plot due the fact that a plane strain, and not plane stress~\citep{zhouIJSS2008} approximation in two dimensions is utilized.  \hsp{What is the normalized value of voltage when $\gamma=100$ for which the electromechanical instability occurs, i.e. how much is this critical voltage increased as compared to the $\gamma=0$ case?}

\begin{figure} \begin{center} 
\includegraphics[scale=0.7]{pics/voltage_hom.pdf}
\caption{Deformation of a single, homogeneously deforming FE subject to voltage loading.}
\label{homo} \end{center} \end{figure}

\subsubsection{Inhomogeneous Deformation}

\begin{figure} \begin{center} 
\includegraphics[scale=0.25]{pics/inhomo3.pdf}
\caption{(a) Schematic of single element with inhomogeneous boundary conditions; (b) Deformed configuration with $\gamma=0$; (c) Deformed configuration with $\gamma=10$.}
\label{inhomo3} \end{center} \end{figure}

Before studying surface tension effects on inhomogeneously deforming DEs, as will be done in subsequent computational examples, we first study a single FE that is fully constrained mechanically along its bottom surface, as illustrated in Figure (\ref{inhomo3})(a).  If no surface tension is present, i.e. $\gamma=0$, the top surface of the element bows outward in response to the applied voltage.  However, for non-zero surface tension $\gamma=10$, the large outward bowing of the top surface that was observed for $\gamma=0$ in Figure (\ref{inhomo3})(b) is reduced drastically, as shown in Figure (\ref{inhomo3})(c).  The effect of the surface tension opposing the deformation that is induced by the voltage is similar to that previously observed for the homogeneous boundary conditions.

\begin{figure} \centering 
\includegraphics[scale=0.7]{pics/voltage_inhom.pdf}
\caption{Deformation of a single, inhomogeneously deforming FE subject to voltage loading.}
\label{inhomo}  \end{figure}

We also analyze the voltage-charge curve for different values of $\gamma$ in Figure (\ref{inhomo}).  As can be seen, the major effect of surface tension is again to significantly increase the value of voltage $\Phi$ that is required in order to induce the voltage-induced electromechanical snap-through instability.

\subsection{Surface Creasing in 2D Strip}

Our next example is based on recent experiments by~\citet{wangPRE2013}, who studied the surface instability mechanisms in constrained DE films in which the top surface is immersed in a fluid.  The fluid was varied such that the top surface of the DE film was subject to different values of surface tension.  Interestingly, upon application of a critical voltage, a transition in the surface instability mechanism from creasing to wrinkling was observed, which was found to be dependent on elastocapillary length $\gamma/(\mu H)$.  Along with the transition in surface instability mechanism, the wavelength of the instability was also found to change, from about $\lambda=1.5H$ for the creasing instability to longer wavelengths, $\lambda=5-12H$ when the elastocapillary length $\gamma/(\mu H)>1$.  

We performed FE simulations of the experiments of~\citet{wangPRE2013}, where the schematic of the problem geometry and the relevant electromechanical boundary conditions is shown in Figure (\ref{film}), i.e. the left, right and bottom surfaces of the film are fixed while a voltage is applied to the top surface, where the surface tension is also present.

\begin{figure} \begin{center} 
\includegraphics[scale=0.7]{pics/film.pdf}
\caption{Schematic of the film with electro-elasto-capillary boundary conditions.}
\label{film} \end{center} \end{figure}

%\hsp{Need here:  (1) Schematic showing the computational setup, including key geometric factors like the height H, the fact that there is surface tension on the top surface, and the boundary conditions.}   The length of the film was taken to be $L=160$, while the height was taken to be $H=4$.  The surface tension $\gamma$ was varied to change the elastocapillary length, while the shear modulus $\mu=1$.  All results are reported for a mesh density of \hsp{XXX}.  

\begin{figure} \begin{center} 
\includegraphics[scale=0.4]{pics/wrinkcreas.pdf}
\caption{Computationally observed transition in the surface instability mechanism in DEs as a function of the elastocapillary length $\gamma/(\mu H)$, for a DE film of dimensions 160x4.  \hsp{In figure a, the displacement magnitude of 6.25 is larger than the film thickness of 4.  That is obviously not physical.}}
\label{wrinkcreas} \end{center} \end{figure}

We show in Figure (\ref{wrinkcreas})(a) the difference in surface instability mechanism depending on the elastocapillary length.  As can be seen, when the surface tension is small, or completely negligible, the surface instability mechanism is that of creasing, or localized folds.  As the elastocapillary length increases to become similar to the film height, the surface instability mechanism changes to wrinkling, as shown in Figure (\ref{wrinkcreas})(b).  The change in instability mechanism is characterized by a significantly larger instability wavelength as compared to the creasing instability in Figure (\ref{wrinkcreas})(a).  Furthermore, rather than abrupt, localized folds as in the creasing instability in Figure (\ref{wrinkcreas})(a), the surface exhibits a more gradual undulating pattern as seen in Figure (\ref{wrinkcreas})(b).  

\begin{figure} \centering 
\includegraphics[scale=0.8]{pics/wave.pdf}
\caption{Normalized instability wavelength as a function of elastocapillary length for 160x4 DE film for two different mesh spacing $e$.  \sss{The agreement with wrinkling theory and our results are not good!}}
\label{wave} \end{figure}

\begin{figure} \centering 
\includegraphics[scale=0.8]{pics/Ec.pdf}
\caption{Normalized electrical field as a function of elastocapillary length for DE film with $L=160$ and $H=4$}  
%\hsp{Need to plot analytic solution together on this plot, remove e=0.5 plot, rename e=1.0 to just FEM solution.}}
\label{wave} \end{figure}

This instability transition was also characterized experimentally by plotting the normalized electric field $E_{c}/\sqrt{\mu/\epsilon}$ and the normalized wavelength $\lambda/H$ as a function of the elastocapillary length~\cite{wangPRE2013}, as shown in Figure (\ref{wave}).  For elastocapillary numbers $\gamma/(\mu H)<1$, we find the normalized wavelength $\lambda/H$ to be close to the value of 1.5 predicted theoretically~\cite{wangPRE2013}.  For larger elastocapillary numbers, i.e. $\gamma/(\mu H)>1$, where wrinkling is observed, a dramatic increase in normalized instability wavelength $\lambda/H$ is observed.  \hsp{The slope of our curve in Figure 9 appears to be different than Figure 4b in the Wang and Zhao 2013 PRE paper, or is it?  Our values for elastocapillary numbers larger than one are smaller than the theory.  You should look through the theory to see if there's any 2D vs. 3D difference.} \sss{In their experiment $\gamma=0.04\ N/m$ but $\gamma/\mu\,H$ is the in the range we used}


\hsp{Need here:  (1) Figure like FIgure 4(a) in the Wang and Zhao PRE paper for the critical electric field vs. elastocapillary length.}

\section{Conclusions}

We have presented a new dynamic, finite deformation finite element model of dielectric elastomers that incorporates surface tension to capture elastocapillary effects on the electromechanical deformation.  

\section{Acknowledgements}

HSP and SS acknowledge funding from the ARO, grant W911NF-14-1-0022.

\section{Appendix}

\begin{figure} \begin{center} 
\includegraphics[scale=0.8]{pics/2d3d.pdf}
\caption{Comparison of 2D and constrained 3D results for inhomogeneous deformation of a single finite element.}
\label{2d3d} \end{center} \end{figure}

We show here verification of the 2D Q1P0 formulation in alleviating volumetric locking.  To do so, we compare it to results obtained using a single 8-node hexahedral element with the previously published Q1P0 formulation of~\citet{parkCMAME2013}.  The 2D results were obtained using the 2D version of the Q1P0 formulation of~\citet{simoCMAME1985}; the boundary conditions on the single 4-node quadrilateral 2D finite element were the same as those previously seen in Figure (\ref{inhomo3})(a).  For the 3D element, in addition to having its bottom surface completely constrained, all $z$-displacements were set to zero to mimic a 2D plane strain problem.  

The results are shown in Figure (\ref{2d3d}).  As can be seen, the 2D and constrained 3D formulations give quite similar results, validating the present 2D formulation.

\bibliographystyle{model2-names}
\bibliography{biball}


\end{document}

%%
%% End of file `elsarticle-template-2-harv.tex'.
